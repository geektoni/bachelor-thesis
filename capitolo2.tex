\chapter{Analisi dei dati}

\section{Wikipedia}

Wikipedia (\textit{https://wikipedia.com}) è un enciclopedia libera online, a contenuto libero e gratuito, lanciata da Jimmy 
Wales e Lerry Sangers nel 2001 ed ora gestita dalla Wikimedia Foundation. Al suo interno sono presenti 45 milioni di voci in 
oltre 280 lingue. Per quanto rigurda questo lavoro, è stata utilizzata la versione italiana di Wikipedia (\textit{http://
it.wikipedia.org}), che al momento (maggio 2017) conta circa 1400000 articoli \cite{wikipedia_stats}. La versione italiana
di Wikipedia nel 2016 ha avuto una media giornaliera di 17 milioni di visite \cite{it_wikipedia_views_2016}.

I dati utilizzati i dati delle page view di Wikipedia \cite{wikipedia_pageviews}. Con page view si intende il numero di visite 
che sono state effettuate su una determinata pagina di Wikipedia in un certo lasso di tempo. I dati disponibili coprono un arco 
di tempo che va dal Dicembre 2007 al Maggio 2016. Essi sono dati grezzi, nel senso che non distinguono tra visite effettuate da 
utenti umani oppure tra visite effettuate da bot. Inoltre, essi sono comprensivi delle visite effettuate tramite dispositivi 
desktop, quindi le visite effettuate da cellulare non sono presenti all'interno delle statistiche.   

\section{Influnet}

I dati riguardanti i livelli di ILI in Italia sono stati ottenuti attraverso l'analisi dei bollettini Influnet
che vengono pubblicati settimanalmente dal Ministero della Salute per tutta la durata del periodo influenzale. 

\section{Analisi dei dati}