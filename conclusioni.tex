\newpage

\chapter{Conclusioni}
\bigskip

La nostra ipotesi iniziale può considerarsi parzialmente confermata da questi esperimenti. 
I risultati espressi precedentemente mostrano come l'analisi e la stima dell'incidenza influenzale in Italia sia possibile.
Da quanto è emerso, durante la stagione influenzale, la popolazione italiana tende a cercare su Wikipedia informazioni 
riguardo ai sintomi e alle patologie ILI. Attraverso l'analisi della variazione del numero di visualizzazioni di certe voci 
selezionate è possibile quindi risalire ad una stima di incidenza del virus influenzale sui cittadini. Questo metodo 
permette anche di produrre delle stime di incidenza ILI in brevissimo tempo, senza incorrere in quelle due-tre settimane di ritardo che i metodi tradizionali di InfluNet comportano. 
\bigskip 

Il modello lineare ha evidenziato come esista questa relazione tra le pageview delle voci di Wikipedia che descrivono 
sintomi e caratteristiche della patologia influenzale, e i dati di incidenza della malattia forniti da InfluNet. Questo 
risultato è interessante poiché è stato ottenuto analizzando un sistema (quello italiano) che presenta sostanziali differenze 
rispetto al sistema delineato dallo studio di riferimento per questo progetto. Infatti, Wikipedia nella versione inglese 
possiede numerose voci legate all'ambito \textit{Influenza} e un volume molto maggiore di traffico sulle varie pagine. 
Inoltre, riguardo ai dati sull'incidenza di ILI, il CDC (\textit{Centre for Disease Control and Prevention}) americano 
fornisce dati sull'incidenza della patologia influenzale che spaziano dalla quarantesima settimana fino alla trentacinquesima 
dell'anno successivo, permettendo di avere un dataset più corposo con cui allenare i vari modelli.      
\bigskip

Si può quindi affermare che Wikipedia rappresenta una fonte di dati affidabile che è possibile 
utilizzare a scopi di ricerca. Il fatto che siano facilmente accessibili (sono rilasciati sotto licenza \textit{Creative 
Commons}) e sempre aggiornati rappresentano alcuni dei punti di forza dei dataset forniti da Wikipedia.   
\bigskip

La validità di questo approccio lo rende idoneo ad essere utilizzato anche per prevedere l'incidenza di altre malattie. 
Sarebbe infatti sufficiente identificare altre voci di Wikipedia, inerenti alla patologia di interesse, e monitorare il 
numero di visite su quelle pagine (oltre ad ottenere dati precisi sull'incidenza della malattia selezionata che dovranno 
essere forniti dagli organi sanitari nazionali o internazionali). Esistono già esempi di ciò, in particolare per malattie 
come ebola, colera, dengue, HIV/AIDS e tubercolosi \cite{Generous2014}.
\bigskip

Questo lavoro sottolinea come sia possibile ottenere importanti informazioni di rilevanza medica attraverso 
l'analisi di fonti di dati non convenzionali. Questo progetto potrebbe fornire, con i 
necessari miglioramenti e dopo aver effettuato ulteriori attività di validazione, un aiuto concreto ai metodi classici 
utilizzati per prevedere l'incidenza influenzale in Italia. Sarebbe possibile anche automatizzare il processo per fornire 
un sistema automatico di sorveglianza dei livelli di patologie influenzali. Il sistema servirebbe da supplemento alle 
metodologie attualmente adottate (come InfluNet) abilitando i vari organi di competenza ad attuare misure di profilassi 
e cura in modo più rapido e mirato, per sopperire efficacemente alle esigenze della sanità pubblica durante le stagioni 
influenzali.
\bigskip  

A differenza della maggior parte dei lavori precedentemente citati, tutti i dati (già formattati ed analizzabili) e il codice 
python utilizzato per questo progetto sono liberamente disponibili in forma di software open source (\url{https://github.com/geektoni/seasonal_influenza_predictor}); in questo modo, 
anche le tecniche e le procedure pratiche per allenare e testare i vari modelli possono essere sottoposte ad analisi e 
critica. Inoltre, grazie alla licenza permissiva utilizzata, tutto il lavoro può essere liberamente scaricato e utilizzato 
anche da altri ricercatori. Ad esempio, i vari script realizzati possono essere anche facilmente adattati per eseguire
analisi sull'incidenza di altre patologie.
\bigskip

Riguardo a sviluppi futuri di questo progetto è giusto segnalare che ci sono comunque diverse migliorie che si potrebbero 
effettuare per renderlo più efficace. Come prima cosa sarà necessario incorporare nel dataset di 
Wikipedia utilizzato per eseguire gli esperimenti i dati riguardanti gli accessi effettuati da dispositivi mobili. 
Come già affermato nei capitoli precedenti, nei dati utilizzati per questo progetto essi non sono stati considerati. 
Fortunatamente, dal maggio 2015 Wikipedia si è attivata per fornire informazioni più rifinite, rispetto a quelle che 
attualmente possiede per l'intervallo 2007-2016. Infatti, essa ha iniziato a offrire dataset sempre più precisi che 
comprendono: le visite effettuate da smartphone/tablet, le visite standard effettuate da desktop e le visite 
effettuate dall'applicazione mobile di Wikipedia. Viene effettuata anche un'attività di filtraggio per eliminare tutto quel 
traffico non umano che è presente sulle voci di Wikipedia (spider, crawlers etc.). L'utilizzo di un dataset più raffinato e 
meno "rumoroso" migliorerebbe sicuramente le prestazioni e ridurrebbe le imprecisione dei modelli. 
\bigskip

Sarebbe anche interessante combinare i dati di Wikipedia con un altra fonte di informazioni per migliorare le capacità 
predittive dei nostri modelli. Ad esempio, i dati delle prescrizioni mediche potrebbero essere di grande aiuto per stimare i 
livelli di incidenza ILI in Italia. Intuitivamente, ad un aumento del consumo di certi farmaci corrisponde anche un aumento 
della popolazione che assume quei farmaci per far fronte a determinati sintomi, ad esempio quelli influenzali. Esistono già 
servizi che permettono di ottenere queste informazioni, ad esempio \textit{OpenPrescribing} 
(\url{https://openprescribing.net/}) permette di ottenere informazioni sulle prescrizioni mediche effettuate in 
Inghilterra. Alcune nazioni invece mettono già a disposizione dati simili; ad esempio, la Danimarca fornisce dati sui 
rimborsi delle prescrizioni mediche ed essi sarebbero una fonte di informazioni ottima per monitorare l'evolversi delle varie 
patologie nel paese \cite{sigrun2012}.
\bigskip