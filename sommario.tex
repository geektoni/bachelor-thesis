\chapter*{Sommario} % senza numerazione
\label{sommario}

\addcontentsline{toc}{chapter}{Sommario} % da aggiungere comunque all'indice

\bigskip
% Caratteristiche generali dell'influenza
L'influenza è un'infezione respiratoria acuta causata principalmente da virus della famiglia \textit{Orthomyxoviridae} 
che possono essere divisi in due ceppi: la varietà A e la varietà B. E' una patologia stagionale
che caratterizza principalmente i mesi invernali (nelle zone a clima temperato) ed è presente in tutto il mondo.
Il vettore di trasmissione principale consiste nelle goccioline di muco e saliva, contenenti il virus, che vengono prodotte
quando una persona infetta starnutisce o tossisce. Questo la rende una malattia di facile e rapida diffusione, 
specialmente nel caso di zone molto affollate. I sintomi riscontrati spesso sono: febbre alta, tosse, emicrania, dolori 
articolari e malessere generale. Normalmente, il decorso dell'infezione è di una settimana; non si ricorre nella maggioranza 
dei casi a particolari cure mediche. In certe categorie a rischio però, se contratta, l'influenza può degenerare e portare 
anche alla morte \cite{whoinfluenza_keyfacts}. 
\bigskip

% Dati generali sulle persone infettate annualmente
Soltanto in Europa, il \textit{Centro Europeo per il controllo delle malattie} (ECDC) indica come l'influenza
stagionale causi da 4 ai 50 milioni di ammalati e circa 15000-70000 morti annuali \cite{ecdc_keyfacts}.
Globalmente, il numero di decessi a causa dell'influenza è di circa da 250000 a 500000 persone all'anno
\cite{whoinfluenza_keyfacts}.  
In Italia l'influenza colpisce mediamente ogni anno l'8\% della popolazione \cite{epicentro}.
Le categorie più colpite sono sopratutto le fasce di popolazione in età pediatrica (0-4 anni e 5-14 anni)
con un incidenza cumulativa che descresce con l'aumentare dell'età. I casi severi e le complicanze
sono più frequenti nei soggetti al di sopra dei 65 anni di età, oppure con condizioni di rischio
rappresentate malattie cardiovascolari, respiratorie o immunitarie \cite{epicentro}.  
\bigskip

% Problemi legati all'influenza e spesa associata
Essendo una patologia che può colpire la maggior parte della popolazione, essa è considerata
un tema di particolare importanza per la sanità pubblica e per la collettività. Si stima che nei paesi industrializzati,
epidemie influenzali possono generare alti livelli di assenteismo, sia lavorativo che scolastico,
e una riduzione della produttività \cite{whoinfluenza_keyfacts}.
In Italia, uno studio ha stimato il costo totale delle epidemie influenzali nel periodo 1999-2008
cha varierebbe da 15 a 20 miliardi di euro \cite{PLLai2011}. Si è anche evidenziato che la durata media 
di assenza dal posto di lavoro a causa dell'influenza è di circa 4,8 giorni. 
Inoltre, i costi diretti in media sono di circa 330 euro a persona (visite, diagnostica, farmaci) e possono salire a circa 
3000-6000 euro in caso di ricovero ospedaliero.
I costi sociali indiretti (inattività scolastica o lavorativa) ammontano invece a 1000 euro a persona \cite{Sessa2005}.
\bigskip

% Storia
Pur essendo una patologia facilmente curabile e prevenibile grazie all'uso degli appositi vaccini, 
l'influenza stagionale è un'infezione che non può essere sottovalutata. Infatti, epidemie e pandemie causate
da questi virus possono essere molto gravi. Si pensi ad esempio alla cosiddetta \textit{influenza spagnola},
una pandemia causata dal virus dell'influenza A sottotipo H1N1 che causò 500 milioni di casi globali e
circa 50 milioni di morti totali tra il 1918 e il 1920 \cite{taubenberger20061918}, addirittura più di quelli
dovuti dalla peste nera del XIV secolo \cite{JAM:JAM1492}.
\bigskip

% Monitoraggio attività influenzali
Attualmente, l'attività dei virus influenzali viene monitorata da alcuni centri facenti
parte del \textit{Global Influenza Surveillance and Response System} (GISRS), un network di sorveglianza globale
sponsorizzato dall'WHO. Questi centri forniscono: informazioni sugli attuali ceppi circolanti, indicazioni
per la produzione dei vaccini antiinfluenzali (su quali varietà focalizzarsi) ed esaminano e conservano
campioni dei virus per scopi di ricerca \cite{whoinfluenza_surveillance}.
\bigskip

Per quanto riguarda l'Italia, il \textit{Centro di Controllo delle Malattie} (CCM) del Ministero della Salute sostiene
che un componente fondamentale per il controllo dell'influenza (sia epidemica che pandemica) sia la sorveglianza.
Nel nostro paese esistono già programmi di monitoraggio dei livelli di ILI (Influenza-Like Illness), come 
InfluNet. Influnet è il sistema nazionale di sorveglianza epidemiologica e virologica; il suo compito è quello di
stimare l'incidenza settimanale della sindrome influenzale (avvalendosi di dati raccolti da medici sentinella disseminati
su tutto il territorio nazionale), in modo da rilevare la durata e l'intensità dell'epidemia \cite{influnet}. Durante la 
stagione invernale vengono pubblicati anche dei bollettini settimanali, tramite il servizio FluNews, che illustrano 
l'evoluzione della situazione italiana \cite{influnet_bollettini}. Questi dati vengono anche condivisi sia con il WHO che con 
ECDC.
\bigskip

% Monitoraggio attivo
InfluNet fornisce informazioni molto importanti relativamente all'incidenza delle patologie influenzali sulla popolazione
italiana; sono tuttavia dati vengono spesso pubblicati con un certo ritardo (in media 2 settimane) rispetto all'arco di tempo 
che descrivono. Per attuare azioni efficaci e coordinare la distribuzione di materiale sanitario, produzione di vaccini etc. 
è necessario avere immediatamente a portata di mano dei dati sulla situazione. 
\bigskip

% Precedenti lavori.
Ci sono stati diversi sforzi per tentare di prevedere o stimare i livelli di incidenza di ILI all'interno della popolazione,
sfruttando fonti di dati non convenzionali (cioè non direttamente le informazioni mediche) \cite{McIver2014, Hickmann2015, 
Generous2014, googleflutrends, Signorini2011}. Normalmente, i dati che vengono sfruttati 
maggiormente sono quelli prodotti dai social media, ad esempio: messaggi di Twitter \cite{Signorini2011}, \textit{page view} 
di Wikipedia \cite{McIver2014, Hickmann2015, Generous2014} e \textit{keywords} di ricerca di Google \cite{googleflutrends}. 
Da questi studi emerge che attraverso l'utilizzo di tecniche di machine learning è possibile arrivare a delle stime dei 
livelli di ILI all'interno della popolazione, con settimane di anticipo rispetto ai metodi tradizionali.
\bigskip

% Obbiettivo tesi
L'obiettivo di questa tesi è replicare, per quanto possibile, alcune parti dei lavori precedentemente citati, per
verificare se anche nel nostro paese sia possibile effettuare, attraverso tecniche di machine learning, un'analisi
attiva per la sorveglianza della diffusione di malattie influenzali. Lo studio che verrà utilizzato come base 
è la ricerca di David J. McIver e John S. Brownstein \cite{McIver2014} in cui vengono delineate delle tecniche per 
l'utilizzo delle \textit{page view} di Wikipedia per stimare il numero di malati settimanali negli Stati Uniti.
\bigskip

% Divisione in capitoli
Il materiale di questo lavoro è suddiviso in capitoli: ognuno di essi tratta una singola parte di tutto il processo svolto. 
Nel primo capitolo viene fornita una descrizione più dettagliata dei dati che sono stati utilizzati in questo progetto
(\textit{page view} di Wikipedia e bollettini di InfluNet). Inoltre, si definiscono i metodi usati per l'analisi degli 
stessi e alcune informazioni statistiche sulla composizione del dataset.
Il secondo capitolo descrive i metodi di machine learning che sono stati selezionati per procedere alla creazione
del modello predittivo finale.
Il terzo capitolo e il quarto capitolo presentano rispettivamente: i risultati dell'esperimento e le riflessioni finali su 
quello che gli esperimenti hanno evidenziato.
\newpage




